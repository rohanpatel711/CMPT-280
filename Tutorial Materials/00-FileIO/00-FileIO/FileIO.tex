\documentclass[cmptslides]{cmpt280-slidesandsolutions}


\date{}
\title{Tutorial}
\subtitle{Java File I/O}

\institute{University of Saskatchewan}
\author{Mark G. Eramian}

\begin{document}

\maketitle

\begin{frame}[fragile]
\frametitle{Opening a File}
\begin{itemize}
	\item In Java, files are abstracted by creating an instance of the \lstinline{java.io.File} class.
	\item It requires the filename as the parameter.
	\item Creating an \lstinline{File} object does not open the file.
\end{itemize}	
\begin{lstlisting}[basicstyle=\tt\small,tabsize=4]
import java.io.File;

public class SomeClass {
	public static void main(String args[]) {
		/** Code snippet to open a file. */
		File myFile = File('filename.txt');
	}
}	
\end{lstlisting}

\end{frame}

\begin{frame}[fragile]
\frametitle{Reading Text Files}
\begin{itemize}
	\item Text files can be opened and read using the \lstinline{java.util.Scanner} class.
	\item An instance of \lstinline{Scanner} can be created from an instance of \lstinline{File}.
\end{itemize}	

\begin{lstlisting}[basicstyle=\tt\small,tabsize=4]
import java.io.File;
import java.util.Scanner;

public class SomeClass {
	public static void main(String args[]) {
		/** Code snippet to open a file. */
		File myFile = File('filename.txt');
		Scanner infile = new Scanner(myFile);
	}
}	
\end{lstlisting}
\end{frame}

\begin{frame}[fragile]
\frametitle{Reading Text Files}
\begin{itemize}
	\item The file is not actually ``opened'' until you instantiate \lstinline{Scanner} so you should use a try-catch block to make sure the file was opened successfully.  For example:
\end{itemize}	

\begin{lstlisting}[basicstyle=\tt\scriptsize,tabsize=4]
import java.io.File;
import java.io.Scanner;
import java.io.FileNotFoundException;

public class SomeClass {
	public static void main(String args[]) {
		Scanner infile = null;
		try {
			infile = new Scanner(new File('filename.txt'));
		}
		catch (FileNotFoundException e) {
			System.out.println("Error: file not found.");
			return;
		}
	}
}	
\end{lstlisting}
\end{frame}


\begin{frame}
	\frametitle{Using \lstinline{Scanner}}
	\begin{itemize}
		\item \textit{Tokens} are sequences of characters from text file that are separated by delimeters.
		\item \lstinline{Scanner} can be used to read the next token from an open file.
		\item The default delimiters are whitespace.
		\item \lstinline{Scanner} has methods to read the next token and convert it to a particular data type.
		\item Example: the \lstinline{Scanner.nextInt()} method reads the next token from the file and tries to convert it to an integer, throwing an exception if that is not possible.
		\item \lstinline{Scanner} also has methods to test whether the next token in the file can be interpreted as a particular data type, e.g. \lstinline{hasNexint()}.
	\end{itemize}
\end{frame}


\begin{frame}[fragile]
\frametitle{Example:  Read a List File of Integers}
\begin{itemize}
	\item Recall from CMPT 141: a list file is a file with one data item (token) per line.
	\item This code below a file with one integer per line and adds them up.
\end{itemize}	
\begin{lstlisting}[basicstyle=\tt\tiny,tabsize=4]
public static void main(String args[]) {
	Scanner infile = null;
	try {
		infile = new Scanner(new File('filename.txt'));
	}
	catch (FileNotFoundException e) {
		System.out.println("Error: file not found.");
		return;
	}
	int sum = 0;
	while( infile.hasNextInt() ) {
		sum = sum + infile.nextInt();
	}
	infile.close()
}	
\end{lstlisting}
\end{frame}


\begin{frame}
\frametitle{Error Checking}
\begin{itemize}
	\item Previous example works as long as every line has a valid number.
	\item If a non-integer is encountered, the file reading stops and later valid integers are not read.
	\item We can use the \lstinline{hasNext()} method, which returns true as long as there is a another token to be read regardless of its interpretation, to distinguish between reaching the end of the file and reaching an invalid token.
	\item This lets us continue adding integers in the remainder of the file even if we encounter a non-integer at some point.
\end{itemize}	
\end{frame}


\begin{frame}[fragile]
\frametitle{Example:  Read a List File with Better Error Checking}
\begin{lstlisting}[basicstyle=\tt\tiny,tabsize=4]
public static void main(String args[]) {
	Scanner infile = null;
	try {
		infile = new Scanner(new File('filename.txt'));
	}
	catch (FileNotFoundException e) {
		System.out.println("Error: file not found.");
		return;
	}
	int sum = 0;
	while( infile.hasNext() ) {
		try {
			sum = sum + infile.nextInt();
		}
		catch( InputMismatchExcepition e ) {
			System.out.println("Warning: non-integer token.");
		}
	}
	infile.close()
}	
\end{lstlisting}
\end{frame}

\begin{frame}
\frametitle{Other Useful Scanner Methods}
\begin{itemize}
	\item \lstinline{hasNextFloat()}, \lstinline{hasNext()}: check for/read float tokens
	\item \lstinline{hasNextLine()}, \lstinline{nextLine()}: check for/read entire lines (without tokenizing).  Useful for reading human-readable text, or when you want to read a line and tokenize it yourself.
	\item \lstinline{next()}:  return the next token, whatever it is (this is how you read string-valued tokens).
	\item \lstinline{useDelimeter(pattern)}: sets the delimeter to exactly \lstinline{pattern}.
	\item Full listing of \lstinline{Scanner} methods in the Java 1.8 docs.
\end{itemize}	
\end{frame}

\begin{frame}
\frametitle{Reading Other File Formats}
\begin{itemize}
\item Recall from CMPT 141 \textit{tabular files}: fixed number of data items per line separated by delimeters; data items on one line may be of different types, but $i$-th data item on a line is always the same type.
\item Two approaches in Java:
	\begin{itemize}[topsep=0pt,itemsep=2pt]
	\item Write logic to read each line, reading each token with appropriate \lstinline{Scanner.nextInt()}, \lstinline{Scanner.nextFloat()}.
	\item Use \lstinline{Scanner.nextLine()} to read each line, use \lstinline{String.trim()} and \lstinline{String.split()} in much the same way as \lstinline{strip()} and \lstinline{split()} are used in python.
	\item In the latter approach, all tokens will be strings, and you have to make the appropriate type conversion yourself after splitting.  This is harder in Java because: no list comprehensions!
	\end{itemize}
\end{itemize}
\end{frame}

\begin{frame}
\frametitle{Exercise 1}
\begin{itemize}
	\item Write java code to read a file where each line contains (in order):
		\begin{itemize}[topsep=0pt]
		\item Month (string)
		\item Year (string)
		\item Min temp (float)
		\item Maz temp (float)
		\end{itemize}
	\item Items on each line are separated by whitespace.
	\item Store the data as a \lstinline{Java.util.ArrayList} of \lstinline{TempRecord} objects (which just stores the four data items and has getters/setters).
	\item Our solution will use the first approach.
\end{itemize}	
\end{frame}

\begin{frame}[fragile]
\frametitle{Writing Files}
\begin{itemize}
	\item There are a mind-wrenching number of ways to write data to a file in Java.
	\item We'll use the \lstinline{BufferedWriter} class.
\end{itemize}	
\begin{lstlisting}[basicstyle=\tt\tiny]
import java.io.FileWriter;
import java.io.IOException;
import java.io.BufferedWriter;

public class WriteFileExample {
    public static void main(String args[]) {
        // Open a file for writing
        BufferedWriter outfile = null;
        try {
            outfile = new BufferedWriter(new FileWriter("filename.txt"));
        }
        catch (IOException e) {
            System.out.println("Error: file cannot be opened.");
            return;
        }
    }	
}
\end{lstlisting}
\end{frame}


\begin{frame}[fragile]
\frametitle{Opening Files for Writing}
\begin{itemize}
	\item There are a mind-wrenching number of ways to write data to a file in Java.
	\item We'll use the \lstinline{BufferedWriter} class.
\end{itemize}	
\begin{lstlisting}[basicstyle=\tt\tiny]
import java.io.FileWriter;
import java.io.IOException;
import java.io.BufferedWriter;

public class WriteFileExample {

    public static void main(String args[]) {
        // Open the file

        BufferedWriter outfile = null;
        try {
            outfile = new BufferedWriter(new FileWriter("stuff.txt"));
        }
        catch (IOException e) {
            System.out.println("Error: file cannot be opened.");
            return;
        }
    }
}	
\end{lstlisting}
\end{frame}

\begin{frame}
\frametitle{Writing data to Files}
\begin{itemize}
	\item The \lstinline{BufferedWriter.write(s)} method will write the string \lstinline{s} to the file opened for writer.
	\item Only strings may be written.  Other data types must be converted before being passed to \lstinline{BufferedWriter.write()}.
	\item All desired whitespace, including newlines, must be explicitly written.
	\item All \lstinline{BufferedWriter.write()} calls have to be in a try-catch block because the method can throw the checked exception \lstinline{IOException}.
	\item You must close the file with the \lstinline{BufferedWriter.close()} method (also must be in try-catch block) or not all of the data you wrote will actually make it to disk.
\end{itemize}
\end{frame}


\begin{frame}[fragile]
\frametitle{Exercise 2}
Write code to write the values of the following variables to a file:
\begin{lstlisting}[basicstyle=\tt\scriptsize]
String s = "The answer to the ultimate question of life,"+
           "the universe, and everything is:";
int fortyTwo = 42;
float fortyTwoPointZero = 42.0;
\end{lstlisting}	
Make sure that each variable's value is on a separate line.
\end{frame}

\begin{frame}
\frametitle{Concluding Remarks}
\begin{itemize}
	\item File I/O in java is much uglier than in Python.
	\item There are other ways to do file I/O beyond what is shown here.  This is just the basics.  
	\item What you've seen here is enough to get you through CMPT 280.
	\item Online tutorials on Java File I/O are numerous and, in the course instructor's opinion, universally awful.
\end{itemize}	
\end{frame}


\end{document}



